\documentclass[a5paper]{book}
\usepackage[utf8]{inputenc}
\usepackage{soulutf8}
\usepackage[russian]{babel}
\usepackage[leqno]{amsmath}
\usepackage{amssymb}
\usepackage{upgreek}
\setcounter{page}{146}
\usepackage[left=2.5cm,top=2.5cm,bottom=2cm]{geometry}
\headsep=10pt
 
\usepackage{fancyhdr}
\pagestyle{fancy}
\fancyhf{}
\fancyhead[LE,RO]{\itshape \thepage}
\fancyhead[CE]{\itshape Глава 10}
\fancyhead[CO]{\itshape Производные высшего порядка}
 
\usepackage{pgfpages}
 
\renewcommand{\headrulewidth}{0.1pt}
\begin{document}
    
   
    {\bfseries \:Теорема 182. }\itshape Если  \upshape 0 < \itshape x \upshape < 1, \itshape то \upshape \par

    $$\lim_{m = \infty} \sum_{\upnu}^{m} {{\frac{1}{2}} \choose{\upnu}} x^{\upnu} = \sqrt{1 + x}.$$

    \so{Доказательство}. Для целых \itshape m \upshape $\geqslant$ 0 имеем \par \par
    \noindent $$\left| {\frac{1}{2} \choose {m + 1}}\right| \;= \; \frac{\left| \underset{k = 0}{\overset{m}\prod} \Big(\frac{1}{2} - \kappa \Big)\right|}{(m+1)!} \; \leqslant \; \frac{ \underset{k = 0}{\overset{m}\prod}\Big(k + \frac{1}{2}\Big)}{(m + 1)!} \;<\;  \frac{\underset{k = 0}{\overset{m}\prod} (k + 1)}{(m+1)!} = 1.$$ \par
   
    %\noindent Я покажу сначала, что для каждого целого $\nu \geqslant 0$
     \noindent \;\;\:Следовательно, в формуле теоремы 181\par
    $$ \left|  {\frac{1}{2} \choose {m + 1}} \frac{x^{m+1}}{y^{m + \frac{1}{2}}} \right| < x^{m+1} \to 0. $$
   
    {\bfseries \:Теорема 183. }\itshapeДля каждого целого m $\geqslant$ 1 и x > 0 \linebreakсуществует\; y\; такое,\; что \upshape \par
    \begin{center}
  $ 1 < y < 1 + x,  $    
   \end{center}
   \begin{center}
    $ \log{}{(1+x)}  =\sum\limits_{\upnu = 1}^{m} {\frac{(-1)^{\upnu - 1}}{\upnu}}x^{\upnu} + {\frac{(-1)^{m}x^{m+1}}{(m+1)y^{m+1}}} $    
   \end{center}

     \so{Доказательство}. Для\par
    $$\hspace{-5mm} f(x) = \log{}{x} \quad(x>0) $$
     имеем\par
    $$\hspace{-5mm} f'(x) = \frac{1}{x}, $$
    и, следовательно (пример 2) к определению 41), для целых\linebreak $\upnu \geqslant 1$ \par
    $$\hspace{-5mm}f^{(\upnu)}(x) = \left({\frac{1}{x}}\right)^{(\upnu - 1)} = {\frac{(-1)^{\upnu - 1}(\upnu - 1)!}{x^{\upnu}}},$$
    $$\hspace{-5mm} {\frac{f^{(\upnu)}(1)}{\upnu!}} = {\frac{(-1)^{\upnu-1}}{\upnu}}. $$
   Следовательно, теорема 177 с $\upxi = 1$, $ h = x$, $n = m + 1$\linebreak обеспечивает сушествование требуемого y.\par 
   \quad \quad \quad \quad\quad \quad\quad\quad\quad\quad\quad\quad\quad\quad\quad\quad\quad\quad\quad\quad



   %\newpage
   \vspace{-1mm}{\bfseries \:Теорема 184. } \itshape Если  \upshape 0 < \itshape x \upshape $\leqslant$ 1, \itshape то \upshape \par
   $$ \hspace{-5mm} \lim_{m = \infty} \sum_{\upnu = 1}^{m}  {\frac{(-1)^{\upnu - 1}}{\upnu}}x^{\upnu} = \log{}{(1+x)}.$$
    \:\:\;\quad\so{Доказательство}. В формуле теоремы 183 имеем:\par
   $$ \left| {\frac{(-1)^{m}x^{m+1}}{(m+1)y^{m+1}}} \right| < \frac{1}{m + 1} \to 0.$$
    \quad \quad \quad \quad\quad \quad\quad\quad\quad\quad\quad\quad\quad\quad\quad\quad\quad\quad\quad\quad
   \begin{center}
   \rule{90pt}{1pt}
   \end{center}
   \quad \quad \quad \quad\quad \quad\quad\quad\quad\quad\quad\quad\quad\quad\quad\quad\quad\quad\quad\quad
    Теорема 183 имеет место для каждого $x > 1$; однако,\linebreak формула теоремм 184 неверна ни для какого $x > 1$, Дей-\linebreak ствительно уже из одного существования
   $$ \hspace{-5mm} \lim_{m = \infty} \sum\limits_{\upnu = 1}^{m} {\frac{(-1)^{\upnu-1}}{\upnu}}x^{\upnu} = \upvarphi(x)$$
   следовало бы, что для целых $m > 1$ при $m \to \infty$ \par
   $$\hspace{-10mm} {\frac{(-1)^{m-1}}{m}}x^{m} =  $$
   $$ \hspace{-5mm} = \sum\limits_{\upnu = 1}^{m}{\frac{(-1)^{\upnu - 1}}{\upnu}}x^{\upnu} - \sum\limits_{\upnu = 1}^{m - 1}{\frac{(-1)^{\upnu - 1}}{\upnu}}x^{\upnu} \to \varphi(x) - \varphi(x) = 0 $$
   и, значит,\par
   $$\hspace{-2mm} \frac{x^m}{m} \to 0, $$
   тогда как, в силу теоремы 180, для целых $m \geqslant 2$ мы имеем\linebreak
  $$ \hspace{-12mm} {\frac{x^m}{m}} = {\frac{(1 + (x - 1))^m}{m}} > {\frac{{m \choose 2} (x - 1)^2}{m}} =  $$
  $$ = (m - 1) {\frac{(x - 1)^2}{2}} \geqslant {\frac{(x - 1)^2}{2}}  \quad(>0). $$
  \;\;\quad10*
  
   
\end{document}






